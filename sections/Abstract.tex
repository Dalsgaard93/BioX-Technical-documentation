\begin{titlepage}
\begin{center}
   \textsc{\LARGE Resumé}\\[1.4cm]  
\end{center}
\textbf{Formål:} Formålet med dette studie er at etablere proof-of-concept for en metode, som kan identificere løbere baseret på data fra kropsbårne sensorer (IMU) og en Fourier-række dreven statistisk AnyBody muskelskeletal løbemodel, som er baseret på 150+ løbere ved brug af principal component analyse og kvadratisk optimering. \textbf{Metode:} To elite løbere er testet på et løbebånd med forskellige selvvalgte hastigheder ($3.89$ og $5.00$ $m/s$). Under testen er motion capture anvendt for at optage kinematisk data \parencite{QualMiqus}. Derudover er accelerationer optaget med fem IMU's placeret ved hhv. sternum i et elastisk pulsrem, højre og venstre håndled med straps samt 5cm over højre og venstre ankel i straps. Alle fem IMUs er tilsvarende repræsenteret i en muskelskeletal 3D model i AnyBody Modeling System som virtuelle accelerometre. De resulterende accelerationer fra de fysiske IMU's bliver omregnet til fourier-række koefficienter, som bruges til at estimere løbestils-relevante parametre ved hjælp af den statistiske løbemodel. Ved brug af kvadratisk programmering sættes koefficienterne fra de virtuelle accelerometer til at være lig med koefficienterne fra IMU sensorerne. Til sidst dannes en fourier-drevet 3D-model af den pågældende løber og løbestil. \textbf{Resultater:} IMU konfigurationerne med enten sternum IMU eller venstre håndled IMU har de højeste Pearson korrelationskoefficienter ($\rho$) mellem de testede konfigurationer sammenlignet med Gold standard, der er baseret på motion capture data. Korrelationerne ved disse varierer mellem en god korrelation $\rho$=$0.73$ til en meget god korrelation $\rho$=$0.98$ med en root mean square error (RMSE), der varierer mellem $2.34$ til $12.47$ $m/s^2$. Forskellen i $\omega$ varierer mellem $-0.11$ to $0.12$ $rad/s$ mellem IMU baseret og GS modeller. Afledt svarer dette til en maksimal forskel i skridt frekvens på 1.15 skridt per minut.
Knæ-, hofte, og glenohumeral fleksion betragtes som relevante parametre for løb, og disse er derfor udvalgt til yderligere analyse og sammenligning mellem de forskellige konfigurationer. Korrelationen for disse varierer fra $\rho$=$0.90$ til $\rho$=$0.99$ med RMSE, der varierer fra $0.03$ til $0.38$ $radianer$ for alle løbere. \textbf{Diskussion:} Dette studie har til formål at etablere proof-of-concept for en metode, der kan bruges til at identificere individuelle løbers løbe-mønstre. Dette sker igennem en pipeline fra rå IMU data i form af accelerationer til en 3D muskoloskeletal model af den pågældende løber. Resultaterne indikerer, at IMU's med en rækkevidde på $\pm$ 16g, placeret på enten sternum eller venstre håndled, er den mest ideelle konfiguration af IMU's for at kunne estimere en løbers bevægelsesmønster.
\vspace{.25cm}\\
\textbf{Dette speciale er fortroligt og udarbejdet i samarbejde med Polar Electro Oy, Kempele}. 
\end{titlepage}
\clearpage

